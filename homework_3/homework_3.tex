\documentclass{article}

\usepackage{fancyhdr}
\usepackage{extramarks}
\usepackage{minted}
\usepackage[T1]{fontenc}
\usepackage{graphicx}
\usepackage{inconsolata}
\usepackage{multicol}
\usepackage{enumitem}


%
% Basic Document Settings
%

\topmargin=-0.45in
\evensidemargin=0in
\oddsidemargin=0in
\textwidth=6.5in
\textheight=9.0in
\headsep=0.25in

\linespread{1.1}

\pagestyle{fancy}
\lhead{\hmwkAuthorName}
\chead{\hmwkClass: \hmwkTitle}
\rhead{\firstxmark}
\lfoot{\lastxmark}
\cfoot{\thepage}

\renewcommand\headrulewidth{0.4pt}
\renewcommand\footrulewidth{0.4pt}

\setlength\parindent{0pt}

%
% Create Problem Sections
%

\newcommand{\enterProblemHeader}[1]{
\nobreak\extramarks{}{Problem \arabic{#1} continued on next page\ldots}\nobreak{}
\nobreak\extramarks{Problem \arabic{#1} (continued)}{Problem \arabic{#1} continued on next page\ldots}\nobreak{}
}

\newcommand{\exitProblemHeader}[1]{
\nobreak\extramarks{Problem \arabic{#1} (continued)}{Problem \arabic{#1} continued on next page\ldots}\nobreak{}
\stepcounter{#1}
\nobreak\extramarks{Problem \arabic{#1}}{}\nobreak{}
}

\setcounter{secnumdepth}{0}
\newcounter{partCounter}
\newcounter{homeworkProblemCounter}
\setcounter{homeworkProblemCounter}{1}
\nobreak\extramarks{Problem \arabic{homeworkProblemCounter}}{}\nobreak{}

%
% Homework Problem Environment
%
% This environment takes an optional argument. When given, it will adjust the
% problem counter. This is useful for when the problems given for your
% assignment aren't sequential. See the last 3 problems of this template for an
% example.
%
\newenvironment{homeworkProblem}[1][-1]{
\ifnum#1>0
\setcounter{homeworkProblemCounter}{#1}
\fi
\section{Problem \arabic{homeworkProblemCounter}}
\setcounter{partCounter}{1}
\enterProblemHeader{homeworkProblemCounter}
}{
\exitProblemHeader{homeworkProblemCounter}
}

%
% Homework Details
%   - Title
%   - Due date
%   - Class
%   - Section/Time
%   - Instructor
%   - Author
%

\newcommand{\hmwkTitle}{Homework\ \#3}
\newcommand{\hmwkDueDate}{February 19, 2016}
\newcommand{\hmwkClass}{Operating Systems}
\newcommand{\hmwkClassTime}{Monday \& Wednesday 3:30pm --- 5:17pm}
\newcommand{\hmwkClassInstructor}{Professor Qu}
\newcommand{\hmwkAuthorName}{Nicholas Land}

%
% Title Page
%

\title{
\vspace{2in}
\textmd{\textbf{\hmwkClass:\ \hmwkTitle}}\\
\normalsize\vspace{0.1in}\small{Due\ on\ \hmwkDueDate\ at 11:59pm}\\
\vspace{0.1in}\large{\textit{\hmwkClassInstructor\ \\ \hmwkClassTime}}
\vspace{3in}
}

\author{\textbf{\hmwkAuthorName}}
\date{}

\renewcommand{\part}[1]{\textbf{\large Part \Alph{partCounter}}\stepcounter{partCounter}\\}

%
% Various Helper Commands
%

% Useful for algorithms
\newcommand{\alg}[1]{\textsc{\bfseries \footnotesize #1}}

% For derivatives
\newcommand{\deriv}[1]{\frac{\mathrm{d}}{\mathrm{d}x} (#1)}

% For partial derivatives
\newcommand{\pderiv}[2]{\frac{\partial}{\partial #1} (#2)}

% Integral dx
\newcommand{\dx}{\mathrm{d}x}

% Alias for the Solution section header
\newcommand{\solution}{\textbf{\large Solution}}

% Probability commands: Expectation, Variance, Covariance, Bias
\newcommand{\E}{\mathrm{E}}
\newcommand{\Var}{\mathrm{Var}}
\newcommand{\Cov}{\mathrm{Cov}}
\newcommand{\Bias}{\mathrm{Bias}}

\begin{document}

  \maketitle

  \pagebreak

  \begin{homeworkProblem}
    \begin{itemize}
      \item Can there be a thread blocked on a semaphore with non-negative value?
      \item Can a semaphore have a negative value without having any threads blocked on it?
    \end{itemize}


    \textbf{\textsc{Solution}} \\

    Yes, because a thread is blocked on 0, and 0 is non-negative. If the value is negative, there are no more resources allocate to use the semaphore. \\

    Yes, because if there are no threads are requesting it than it can have a negative value.
  \end{homeworkProblem}

  \begin{homeworkProblem}
    In the following code, four processes produce output using the routine `printf' and synchronize using three semaphores `R', `S' and `T.' \underline{We assume function `printf' wont cause context switch.}

    \begin{minted}{C++}
      Semaphore R=1, S=3, T=0; /* Initialization */
    \end{minted}

    \begin{multicols}{4}
      \begin{minted}{C++}
        /* Process 1 */
        while(true) {
          P(S);
          printf('A');
        }
      \end{minted}

      \vfill
      \columnbreak

      \begin{minted}{C++}
        /* Process 2 */
        while(true) {
          P(T);
          printf('B');
          printf('C');
          V(T);
        }
      \end{minted}

      \vfill
      \columnbreak

      \begin{minted}{C++}
        /* Process 3 */
        while(true) {
          P(T);
          printf('D');
          V(R);
        }
      \end{minted}

      \vfill
      \columnbreak

      \begin{minted}{C++}
        /* Process 4 */
        while(true) {
          P(R);
          printf('E');
          V(T);
        }
      \end{minted}
    \end{multicols}

    \begin{enumerate}[label=\alph*)]
      \item How many \textbf{A}\textquotesingle s and \textbf{B}\textquotesingle s are printed when this set of processes runs?

      \item What is the smallest number of \textbf{D}\textquotesingle s that might be printed when this set of processes runs?

      \item Is \textbf{AEBCBCDAA} a possible output sequence when this set of processes runs? Clarify your answer.
    \end{enumerate}

    \textbf{\textsc{Solution}} \\

    \begin{enumerate}[label=\alph*)]
      \item Three \textbf{A}\textquotesingle s are printed because S is decremented, but it is never incremented.

      B can be printed 0, 1, or \{B\}* times. It is possible that be could be infinite. However, It is also possible that process 3 and process 4 could be in an infinite loop, and in that case B would not be printed. It could also be that B could get printed only one time, and then process 3 and 4 are in an infinite loop. B could be printed \{B\}* times if process 2 was in an infinite loop.

      \item 0 times. If process 4 is run, it is possible that process 2 could run, and then go between process 2 and 4 infinitely so long as there is no waiting queue.

      If there is a waiting queue then D can be printed infinitely many times.

      \item Yes, because processes 1 could run, then process 4, then process 2, then process 2, then process 2, then process 2, then process 2, then process 3, then process 1, and finally process 1 if there is a waiting queue.

      If there is a waiting queue then this would not be a possible output, because then process 2 is not able to be in the waiting queue when it is being executed. So the output BCBC would be unable to complete.
    \end{enumerate}
  \end{homeworkProblem}

  \pagebreak

\end{document}
