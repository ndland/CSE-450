\documentclass{article}

\usepackage{fancyhdr}
\usepackage{extramarks}
\usepackage{amsmath}
\usepackage{amsthm}
\usepackage{amsfonts}
\usepackage{tikz}
\usepackage[plain]{algorithm}
\usepackage{algpseudocode}
\usepackage{minted}


%
% Basic Document Settings
%

\topmargin=-0.45in
\evensidemargin=0in
\oddsidemargin=0in
\textwidth=6.5in
\textheight=9.0in
\headsep=0.25in

\linespread{1.1}

\pagestyle{fancy}
\lhead{\hmwkAuthorName}
\chead{\hmwkClass: \hmwkTitle}
\rhead{\firstxmark}
\lfoot{\lastxmark}
\cfoot{\thepage}

\renewcommand\headrulewidth{0.4pt}
\renewcommand\footrulewidth{0.4pt}

\setlength\parindent{0pt}

%
% Create Problem Sections
%

\newcommand{\enterProblemHeader}[1]{
    \nobreak\extramarks{}{Problem \arabic{#1} continued on next page\ldots}\nobreak{}
    \nobreak\extramarks{Problem \arabic{#1} (continued)}{Problem \arabic{#1} continued on next page\ldots}\nobreak{}
}

\newcommand{\exitProblemHeader}[1]{
    \nobreak\extramarks{Problem \arabic{#1} (continued)}{Problem \arabic{#1} continued on next page\ldots}\nobreak{}
    \stepcounter{#1}
    \nobreak\extramarks{Problem \arabic{#1}}{}\nobreak{}
}

\setcounter{secnumdepth}{0}
\newcounter{partCounter}
\newcounter{homeworkProblemCounter}
\setcounter{homeworkProblemCounter}{1}
\nobreak\extramarks{Problem \arabic{homeworkProblemCounter}}{}\nobreak{}

%
% Homework Problem Environment
%
% This environment takes an optional argument. When given, it will adjust the
% problem counter. This is useful for when the problems given for your
% assignment aren't sequential. See the last 3 problems of this template for an
% example.
%
\newenvironment{homeworkProblem}[1][-1]{
    \ifnum#1>0
        \setcounter{homeworkProblemCounter}{#1}
    \fi
    \section{Problem \arabic{homeworkProblemCounter}}
    \setcounter{partCounter}{1}
    \enterProblemHeader{homeworkProblemCounter}
}{
    \exitProblemHeader{homeworkProblemCounter}
}

%
% Homework Details
%   - Title
%   - Due date
%   - Class
%   - Section/Time
%   - Instructor
%   - Author
%

\newcommand{\hmwkTitle}{Homework\ \#1}
\newcommand{\hmwkDueDate}{January 18, 2016}
\newcommand{\hmwkClass}{Operating Systems}
\newcommand{\hmwkClassTime}{Monday \& Wednesday 3:30pm --- 5:17pm}
\newcommand{\hmwkClassInstructor}{Professor Qu}
\newcommand{\hmwkAuthorName}{Nicholas Land}

%
% Title Page
%

\title{
    \vspace{2in}
    \textmd{\textbf{\hmwkClass:\ \hmwkTitle}}\\
    \normalsize\vspace{0.1in}\small{Due\ on\ \hmwkDueDate\ at 11:59pm}\\
    \vspace{0.1in}\large{\textit{\hmwkClassInstructor\ \\ \hmwkClassTime}}
    \vspace{3in}
}

\author{\textbf{\hmwkAuthorName}}
\date{}

\renewcommand{\part}[1]{\textbf{\large Part \Alph{partCounter}}\stepcounter{partCounter}\\}

%
% Various Helper Commands
%

% Useful for algorithms
\newcommand{\alg}[1]{\textsc{\bfseries \footnotesize #1}}

% For derivatives
\newcommand{\deriv}[1]{\frac{\mathrm{d}}{\mathrm{d}x} (#1)}

% For partial derivatives
\newcommand{\pderiv}[2]{\frac{\partial}{\partial #1} (#2)}

% Integral dx
\newcommand{\dx}{\mathrm{d}x}

% Alias for the Solution section header
\newcommand{\solution}{\textbf{\large Solution}}

% Probability commands: Expectation, Variance, Covariance, Bias
\newcommand{\E}{\mathrm{E}}
\newcommand{\Var}{\mathrm{Var}}
\newcommand{\Cov}{\mathrm{Cov}}
\newcommand{\Bias}{\mathrm{Bias}}

\begin{document}

\maketitle

\pagebreak

\begin{homeworkProblem}
  A computer is built that uses 15 bits for integers and for addresses.

    \begin{enumerate}
        \item How many bytes of memory can be addressed?
        \item What is the range of values for signed (two's complement) integers.
        We want the largest magnitude negative number, and the largest magnitude
        positive number.
    \end{enumerate}

    \textbf{Solution}

    \begin{enumerate}
      \item $2^{15} = 32,768$ bits can be addressed, which is $\frac{32768}{8} = 4,096$ bytes.
      \item $-2^{14} = -16,384$ is the samllest negative integer \& $2^{14} - 1
      = 16,383$ is the largest integer.
    \end{enumerate}

\end{homeworkProblem}

\begin{homeworkProblem}
  Describe two of the primary motivations for having Virtual Memory in the computer system. \\

  \textbf{Solution}

  \begin{enumerate}
    \item To allow efficient and safe sharing of memory among multiple programs.
    \item To allow a user to exceed the size of primary memory.
  \end{enumerate}
\end{homeworkProblem}

\begin{homeworkProblem}
  Explain why virtual memory in a system without Translation Lookaside Buffer will be much slower than physical memory. \\

  \textbf{Solution} \\

  If a system did not contain a Translation Lookaside Buffer it would be much slower than physical memory because there would be an extra step involved in looking a page address up. Because on every reference the virtual page number is looked up in TLB, and without a TLB, it would result in a miss. A TLB miss takes about 13 clock cycles.
\end{homeworkProblem}

\begin{homeworkProblem}
  Describe what is wrong with the following function and propose modifications to fix it, submit with a test program and test results.
  \begin{minted}[linenos]{C++}
char *string_duplicator(char *s) {
        char *cpy;
        for (int i = 0; i <= strlen(s); i++) {
                cpy[i] = s[i];
        }
        return(cpy);
}
\end{minted}

  \textbf{Solution} \\

  The problem with the previous function was two fold; the first issue was `strlen` was part of a library that was not imported. The second issue was that the variable 'i' was not declared. \\

  \textbf{Proposed Solution}

  \begin{minted}[linenos]{C++}
#include <stdio.h>
#include <string.h>

// Prototype so that the main function can call
// string_duplicator
char *string_duplicator(char *s);

int main(int argc, char const *argv[]) {
        char testString[] = "hello";
        printf("%s\n", string_duplicator(testString));
}

char *string_duplicator(char *s) {
        char *cpy = new char;
        for (int i = 0; i <= strlen(s); i++) {
                cpy[i] = s[i];
        }
        return(cpy);
}
  \end{minted}

  This produces a result of \textbf{Hello}.
\end{homeworkProblem}

\begin{homeworkProblem}
  Please submit a testing program and test results together with a brief readme file describing what the main function of the function test.

  \begin{minted}[linenos]{C++}
int test(char *s) {
        int x = 0;
        unsigned char c; while (*s) {
                c = *s; while (c) {
                        if (c & 1) x++;
                        c = c >>1;
                }
                s++;
        }
        return x;
}
  \end{minted}

  \textbf{Explanation of test function} \\

  The above function loops through each character of 's' and checks to see if there is a '1' stored in the address of that character. If there is a one, it increments x by 1, if there is no one, it bitshifts right by 1. \\

  \textbf{Proposed Solution}

  \begin{minted}[linenos]{C++}
int test(char *s) {
        int x = 0;
        unsigned char c; while (*s) {
                c = *s; while (c) {
                        if (c & 1) x++;
                        c = c >>1;
                }
                s++;
        }
        return x;
}

int main(int argc, char const *argv[]) {
        char testString[] = "Hello, World";
        printf("%d\n", test(testString));
}
  \end{minted}

This produces a result of \textbf{46}.
\end{homeworkProblem}

\end{document}
