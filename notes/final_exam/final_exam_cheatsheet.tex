\documentclass{article}

\usepackage[margin=0.5in]{geometry}
\usepackage{multicol}
\usepackage[T1]{fontenc}
\usepackage{lipsum}
\usepackage{color}
\usepackage{tikz}
\usepackage{amsmath}
\usepackage[usenames, dvipsnames]{color}
\usepackage[linesnumbered,boxed,commentsnumbered,lined]{algorithm2e}
\usetikzlibrary{positioning}

\tikzset{
  gray box/.style={
    fill=gray!20,
    draw=gray,
    minimum width={2*#1ex},
    minimum height={2em},
  },
  annotation/.style={
    anchor=north,
  }
}

\setlength\parindent{0pt}

% Alias for bold small caps
\newcommand{\smallcaps}[1]{\textsc{\textbf #1}\\}

\begin{document}
\begin{multicols}{2}
  A computer whose processes have 1024 pages in their address spaces keeps
  its page tables in memory. The overhead required for reading a word from
  the page table is 500 nsec.  To reduce this  overhead, the computer has an
  associative  memory  which holds 32 (virtual  page,  physical  page frame)
  pairs and can do look up in 100 nsec.  What hit rate is needed  to reduce
  the mean overhead to 200 nsec?

  \smallcaps{Solution}

  The effective instruction time is $100h + 500(1-h)$, where h is the
  hit rate. $100$ is the look up time \& $500$ is the overhead. Set the
  expression equal to 200 and solve for h. We get h must be 0.75 (or greater).

  \begin{align*}
    100h + 500(1-h) &= 200 \\
    100h + 500 - 500h &= 200 \\
    -400h + 500 &= 200 \\
    -400h &= -300 \\
    h &= \frac{-300}{-400} \\
    h &= \frac{3}{4} \implies 75\%
  \end{align*}

  \section*{Address Translation}
  \begin{tabular}{|c|c|c|}
    \hline
    Frame Number & Process ID & Page Number \\
    \hline
    0 & 1 & 2 \\
    \hline
    1 & 1 & 1 \\
    \hline
    2 & 2 & 1 \\
    \hline
    3 & 3 & 0 \\
    \hline
    4 & 1 & 3 \\
    \hline
  \end{tabular}

  Using the table above, translate the following:

  \begin{enumerate}
    \item To which physical address does virtual address 130 of process 1 map to?
      If this virtual address does not map to any physical address, write `does not map'.
    \item To which physical address does virtual address 17 of process 2 map to?
      If this virtual address does not map to any physical address, write `does not map'.
    \item Which virtual address of which process maps to physical address 50?
  \end{enumerate}

  \smallcaps{Solution}

  \begin{enumerate}
    \item logical $= (1 \times 100) + 30 \implies 130$ \\
      Physical $=$ frame\# $= 1 \implies (1 \times 100) + 30 = 130$
    \item Virtual address 17 does not map
    \item Physical address $= (0 \times 100) + 50 = 50$ Where $0$ is the frame\# \\
      So, logical address $= (2 \times 100) + 50 = 250$ Where $2$ is the page\#
  \end{enumerate}

  \section*{File system implementation}
  Consider a File system that maintains unique index node for each
  file in the system. Each index node includes 10 direct pointers, a single indirect pointer,
  and a double indirect pointer. The file system block size is 1024 bytes, and a block pointer
  occupies 4 bytes.

  \begin{enumerate}
    \item What is the maximum file size that can be supported by the index node?
    \item How many disk operations will be required if a process read data from the $N^{th}$ block
      of a file? Assume that the file is already open, the buffer cache is empty, and each disk
      operation read a single file block. Your answer should be given in terms of $N$.
  \end{enumerate}

  \smallcaps{Solution}

  \begin{enumerate}
    \item $1024 * (10 + 2^8 + 2^8 * 2^8)$ \\
      $\implies 2^{10} * (2^3 + 2 + 2^8 + 2^{16})$ \\
      $\implies 2^{13} + 2^{11} + 2^{18} + 2^{26}$ \\
    \item $0 \le N < 8$, One operation \\
      $8 \le N < 256 + 8$, Two operations \\
      $256 + 8 \le N < 2^{13} + 2^{11} + 2^{18} + 2^{26}$, Three operations
  \end{enumerate}

\end{multicols} \end{document}
