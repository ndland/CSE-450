\documentclass[titlepage]{article}
\usepackage{datetime}
\usepackage[usenames, dvipsnames]{color}
\usepackage{algorithm2e}


\newcommand*{\redbold}[1]{\textbf{{\color{red} #1}}}
\newcommand*{\redit}[1]{\textit{{\color{red} #1}}}
\newcommand*{\bluebold}[1]{\textbf{{\color{blue} #1}}}
\newcommand*{\orangebold}[1]{\textbf{{\color{Orange} #1}}}

\title{Exam Notes}
\author{Nicholas Land}
\date{\formatdate{23}{02}{2016}}

\begin{document}
  \maketitle

  \centering{\section*{Processes}}

  \centering{\section*{Process \& Thread Synchronization}}

  \paragraph{Background}

  \begin{itemize}
    \item Parallelism can provide a distinct way of conceptualizing problems
    \item Concurrent access to shared data may result in data inconsistency
    \item Maintaining data consistency requires mechanisms to ensure that the orderly exectution of cooperating processes
    \item Suppose that we wanted to provide a solution to the consumer-producer problem that fills \redbold{all} the buffers

    \begin{itemize}
      \item We can do so by having an integer \redit{count} that keeps track of the number of full buffers
      \item Initially the \redbold{count} is set to $0$
      \item It is incremented by the producer after it produces a new buffer
      \item It is decremented by the consumer after it consumers a buffer
    \end{itemize}
  \end{itemize}

  \paragraph{Race Condition}

  Race conditions can occur when two operations on shared variables are not \textbf{atomic}

  \paragraph{Definitions}

  \begin{itemize}
    \item \bluebold{Synchronization} using atomic operations to ensure cooperation between threads
    \item \bluebold{Critical Section} piece of code that only one thread can execute at once. Only one thread at a time will get into this section of code
    \begin{itemize}
      \item \orangebold{Mutual Exclusion} ensuring that only one thread does a particular thing at a time
      \item \orangebold{Progress} selecting a thread to enter cannot postpone indefinitely
      \item \orangebold{Bounded Waiting} before entering the critical section
    \end{itemize}
  \end{itemize}

  {\color{blue} Important idea: all synchronization involves waiting}


  \paragraph{Peterson's Solution}

  \begin{itemize}
    \item A solution for two processes
    \item Assum that the LOAD and STORE instructions are atomic
    \begin{itemize}
      \item \textbf{Atomic} == cannot be interrupted
    \end{itemize}
    \item The two processes share two variables
    \begin{itemize}
      \item int {\color{red} turn}
      \item Boolean {\color{red} flag[2]}
    \end{itemize}
    \item The variable {\color{red} turn} indicates whos turn it is to enter the critical section
    \item the {\color{red} flag} array is used to indicate if a process is ready to enter the critical section
    \begin{itemize}
      \item {\color{red} flag[i] = true} implies that {\color{blue} $P_1$} is ready
    \end{itemize}
  \end{itemize}

  \paragraph{Alogrithm for Process {\color{blue} $P_1$}}

  \begin{center}
    \begin{algorithm}
      \SetAlgoLined
      \While{true}{
        flag[i] = TRUE\;
        turn = j\;
        \While{flag[j] \&\& turn == j}{
          \vdots
          CRITICAL SECTION\;
          flag[i] = FALSE\;
          REMAINDER SECTION\;
        }
      }
      \caption{Peterson's Solution}
    \end{algorithm}
  \end{center}

  \paragraph{Semaphore}

  \begin{itemize}
    \item Synchronization tool that does not require busy waiting
    \begin{itemize}
      \item integer variable
      \item Two standard operations
      \begin{itemize}
        \item {\color{blue} S.wait()} $\Rightarrow$ P()
        \item {\color{blue} S.signal()} $\Rightarrow$ V()
      \end{itemize}
      \item Less complicated
    \end{itemize}
    \item Can only be accessed via two indivisible (atomic) operations
  \end{itemize}

  \paragraph{Semaphore as General Synchronization Tool}

  \begin{itemize}
    \item {\color{red} Counting} semaphore -- integer value can range over unrestricted domain
    \item {\color{red} Binary} semaphore -- integer value that can range only between 0 \& 1; can be simpler to implement
    \begin{itemize}
      \item Also known as {\color{red} mutex locks}
    \end{itemize}
    \item Can implement a counting semaphore {\color{blue} S} as a binary semaphore
    \item Provides mutual exclusion
  \end{itemize}

\end{document}
